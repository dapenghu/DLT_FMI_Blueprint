% !TEX root = ./main.tex
\section{金融市场基础设施原则与分布式账本}\label{sec:pfmi}

2008年金融危机爆发以来,国际社会对于构建安全、高效的金融基础设施提出了更高的要求。十多年来诞生了两方面的成果。

一个是2012年国际清算银行(BIS)和国际证监会组织(IOSCO)联合发布的《金融市场基础设施原则》(PFMI)\cite{pfmi}。
针对金融市场基础设施的建设,PFMI分析了金融系统的主要风险(系统性风险、法律风险、流动性风险、业务风险、托管与投资风险、运营风险),
从9个方面(总体架构、信用与流动性风险管理、结算、证券存管与交易、违约管理、运营风险管理、准入机制、效率、透明度)提出了24条指导原则。
PFMI从金融系统顶层设计的角度,进一步确立了全球金融市场基础设施建设标准,为世界各国开展相关工作指明了方向。

另一个是以比特币为代表的分布式账本技术(DLT)。
在2008年,中本聪(Satoshi Nakamoto)发表了比特币的白皮书,描述了一种不依赖于单一信用机构的记账技术。
这种后来被称为分布式账本的新技术迅速发展,各种数字货币与公有链竞相出现,构建了一个全新的金融服务生态,引起国际社会特别是金融界的广泛关注和高度重视。
分布式账本提供了公共的账本管理平台,互相独立的金融机构可以在同一个账本上彼此协作。
这种公共账本模式为对金融系统的深远影响体现在以下几个方面:

\begin{itemize}
    \item[\dag] \textbf{效率}:
    跨行、跨境交易可以在公共账本上一次性完成结算,缩短了交易环节,简化了交易流程,降低了交易摩擦。

    \item[\dag] \textbf{风险}:
    由于减少了金融中介,降低了中间人的违约风险、信用风险、流动性风险。

    \item[\dag] \textbf{透明度}:
    资金流数据集中记录在一个公开、透明的账本上,提高了交易数据可访问性。
    相对于监管多个分散的账本,监管一个公共账本会大幅降低监管的难度。
    无论是对于监管层、还是金融机构来讲,监管合规的成本也会大幅降低。
    
    \item[\dag] \textbf{自动化}:
    利用智能合约,将金融资产映射为可编程的Token,提高资产发行、销售、流转、托管等业务的电子化、自动化。
    
    \item[\dag] \textbf{标准化}:
    众多金融机构使用共享的账本管理系统,便于统一通信协议、数据格式,提高金融行业规范化、标准化程度。
    
    \item[\dag] \textbf{全球化}:
    分布式账本建立在全球化的互联网基础上,而不是封闭的专用金融网络。
    各个国家和地区可以共享同一套系统,便于搭建跨境的金融基础设施。
    
    \item[\dag] \textbf{避免重复建设}:
    采用国际化的公共账本,对于金融机构,可以节省维护独立账本的负担;
    对于国家和地区,节省了各自建设金融专用网络的成本。
\end{itemize}

总之,对于金融业的未来发展,PFMI和分布式账本都有重大的价值。
PFMI 是自顶向下、为金融系统的顶层设计提供了指导原则;
分布式账本是自底向上、在全新的账本管理技术的基础上,重构整个金融系统。
PFMI与分布式账本技术的出现有相同的时代背景,都是源自于次贷危机的影响;
它们的目标都是要建立更加安全、高效、可靠、可信的金融系统。
二者虽然方法和方式不同,但是殊途同归,本质上是互补的。
借助于分布式账本技术,可以降低PFMI的执行成本,有利于PFMI的推广;
反之,坚持PFMI的指导原则,可以吸取前人的经验教训,有利于安全、稳妥地将分布式账本技术应用于金融基础设施建设。

