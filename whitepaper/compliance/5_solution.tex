% !TEX root = ./main.tex
\section{组织安排}\label{sec:arrangement}
根据第3节的结论,单纯的分布式账本系统满足了隐私性和去中心化,但是牺牲了完整性;
单纯的中心化账本管理系统能够同时满足完整性和隐私性,但是牺牲了去中心化。
基于分布式账本的金融基础设施必须要创造性地与中心化账本系统结合,才能满足监管合规的要求。

我们的解决方案基于这样一个发现:交易的合规审核与结算是前后两个不同的阶段,可以由不同的参与者在不同的数据上执行。
在合规审核阶段,需要完整的、真实的身份信息、交易背景信息等隐私数据;
但是在结算阶段,只需要检查账户余额等非隐私数据。
审核和结算是可以分离的,合规审核适合在中心化账本系统里执行,在不泄露用户的隐私数据的前提下,完成反洗钱等审核;
结算适合在分布式账本里执行,处理的交易数据是经过脱敏的,不需要担心隐私问题。
这样中心化账本和分布式账本集成在一起,最大限度的整合二者的优点。

%根据这个总体设计思路,下面从组织安排、账本数据结构、合规验证协议、智能合约、准入规则等多个角度详细阐述 Libra 合规架构设计。

在传统金融系统中,银行等金融机构接受用户的委托,以负债的形式托管用户的资产,同时拥有记账权、审核权。
也就是说,银行不但负责代理用户处理贷记/借记事务,同时还要管理用户的隐私信息,以及根据金融监管的要求审核每一笔交易的合规性。

在 Libra 生态中,记账权与审核权是分离的。
用户的资产记录在 Libra 分布式账本上,矿工拥有记账权,负责账本的管理。
金融机构的角色也发生变化,它们失去了记账权,依然拥有审核权,负责管理、保护用户的隐私数据,并且审核每一笔交易。
这种只负责合规审核、不负责记账的新金融机构称之为\textbf{合规机构}。
它为 Libra 分布式账本提供预言机服务,连接链上的交易数据和链下的背景数据,保证账本数据的完整性。

合规机构作为一种新的角色,可以是原来的银行、保险公司、交易所等金融机构,也可以由监管机构直接负责。
合规机构是中心化的,需要信用背书,它必须持是持有相关牌照的经营实体,接受当地司法辖区的监管,能够有效的保护用户的隐私、行使合规服务的相关责任。

从用户的使用体验来讲,数字资产记录在 Libra 分布式账本上,通过控制账户对应的私钥确立资产的所有权,与其它公链差异不大。
所有资产的处置都必须通过对交易做数字签名才能执行。
不同于比特币、以太坊等其它分布式账本,Libra 的账户有注册机制。
每一个用户都需要向当地辖区的一个合规机构注册自己的真实身份,审核通过之后,在合规机构的协助下建立 Libra 账户。
只有合规机构知晓账户对应的真实身份信息,对于其它人是保密的。
每一笔交易也都需要合规机构的参与,验证交易是否合规。

从监管层的角度来讲,合规机构负责保证 Libra 平台上资产和交易背景数据的完整性、真实性、有效性。是执行反洗钱、反恐怖融资等相关金融政策的主体。
通过对合规机构的监督和管理,能够实现去中心化金融系统上的中心化监管,降低监管成本,有效的推动相关金融政策的实施,维护金融稳定。

从 Libra 协会的角度来讲,合规机构是 Libra 构建金融基础设施的重要组成部分,是连接用户、金融监管系统的桥梁,是社区治理的重要枢纽。
一方面,合规机构借助于 Libra 网络拓展用户群体,提供支付、借贷、证券等合规金融服务;
另一方面,Libra 通过合规机构解释并执行各个辖区的法律法规,监控、抵制在 Libra 系统上的金融犯罪。
Libra 协会为合规机构提供技术支持与相关咨询服务。
由于合规机构必须是有信用背书的法人实体,也必须是实名制的。
在开展业务之前,合规机构需要向 Libra 协会注册。根据当地辖区的法律要求,Libra 协会审核合其资格与经营范围,然后授权其在 Libra 网络上开展业务。
由于合规机构的特殊性和重要性,Libra 协会为合规机构发放 X.509 数字证书,任何人可以通过此证书验证它的真实性,防止被钓鱼欺诈。




